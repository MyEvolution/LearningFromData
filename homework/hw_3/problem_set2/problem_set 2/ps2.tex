% Homework template for Learning from Data
% by Xiangxiang Xu <xiangxiangxu.thu@gmail.com>
% LAST UPDATE: October 8, 2018
\documentclass[a4paper]{article}
\usepackage[T1]{fontenc}
\usepackage{amsmath, amssymb, amsthm}
% amsmath: equation*, amssymb: mathbb, amsthm: proof
\usepackage{moreenum}
\usepackage{mathtools}
\usepackage{url}
\usepackage{enumitem}
\usepackage{bm}
\usepackage{graphicx}
\usepackage{subcaption}
\usepackage{booktabs} % toprule
\usepackage[colorlinks,linkcolor=blue]{hyperref}
\usepackage[mathcal]{eucal}
\usepackage{dsfont}
\usepackage[numbered,framed]{matlab-prettifier}
%% Definitions for Learning from Data
%% UPDATE: October 8, 2018 by Xiangxiang 
\newcommand{\theterm}{Fall 2018}

\newcommand{\thecoursename}{
Tsinghua-Berkeley Shenzhen Institute\\
%\vspace*{0.1in}
\textsc{Learning from Data}
}

\newcommand{\courseheader}{
\vspace*{-1in}
\begin{center}
\thecoursename \\
\theterm
\vspace*{0.1in}
\hrule
\end{center}
}
\newcommand{\uc}{\underline{c}}    % c, vec
\newcommand{\uv}{\underline{v}}    % x, vec
\newcommand{\uw}{\underline{w}}    % w, vec
\newcommand{\ux}{\underline{x}}    % x, vec
\newcommand{\uy}{\underline{y}}    % y, vec
\newcommand{\uz}{\underline{z}}    % z, vec
\newcommand{\um}{\underline{m}}    % m, vec
\newcommand{\ut}{\underline{t}}    % t, vec
\newcommand{\bc}{\bm{c}}    % c, vec
\newcommand{\bv}{\bm{v}}    % x, vec
\newcommand{\bw}{\bm{w}}    % w, vec
\newcommand{\bx}{\bm{x}}    % x, vec
\newcommand{\by}{\bm{y}}    % y, vec
\newcommand{\bz}{\bm{z}}    % z, vec
% \newcommand{\bm}{\bm{m}}    % m, vec
\newcommand{\bt}{\bm{t}}    % t, vec

\newcommand{\balpha}{\bm{\alpha}}    % alpha, vec
\newcommand{\bxi}{\bm{\xi}}    % xi, vec


\newcommand{\rvx}{\mathsf{x}}    % x, r.v.
\newcommand{\rvy}{\mathsf{y}}    % y, r.v.
\newcommand{\rvz}{\mathsf{z}}    % z, r.v.
\newcommand{\rvw}{\mathsf{w}}    % w, r.v.
\newcommand{\rvv}{\mathsf{v}}    % v, r.v.
\newcommand{\rvm}{\mathsf{m}}    % m, r.v.
\newcommand{\rvt}{\mathsf{t}}    % t, r.v.
\newcommand{\rvH}{\mathsf{H}}    % H, r.v.
\newcommand{\urvx}{\underline{\mathsf{x}}}    % x, r.v. vec
\newcommand{\urvy}{\underline{\mathsf{y}}}    % y, r.v. vec
\newcommand{\urvz}{\underline{\mathsf{z}}}    % z, r.v. vec
\newcommand{\urvw}{\underline{\mathsf{w}}}    % w, r.v. vec
\newcommand{\urvt}{\underline{\mathsf{t}}}    % t, r.v. vec
\newcommand{\defeq}{\triangleq} %\coloneqq
\newcommand{\reals}{\mathbb{R}}
\newcommand{\T}{\mathrm{T}}    % transpose
\newcommand{\BLS}{\mathrm{BLS}}    % BLS
\newcommand{\LLS}{\mathrm{LLS}}    % LLS
\newcommand{\MVU}{\mathrm{MVU}}    % MVU

\DeclareMathOperator*{\maximize}{maximize}    % maximize
\DeclareMathOperator*{\minimize}{minimize}    % minimize
\newcommand{\st}{\mathrm{subject~to}}    % minimize


% \newcommand{\E}[1]{\mathbb{E}\left[{#1}\right]}
% \newcommand{\Prob}[1]{\mathbb{P}\left({#1}\right)}
\DeclareMathOperator*{\argmax}{arg\,max}
\DeclareMathOperator*{\argmin}{arg\,min}
\DeclareMathOperator*{\argsup}{arg\,sup}
\DeclareMathOperator*{\arginf}{arg\,inf}
\DeclareMathOperator{\Var}{Var}
\DeclareMathOperator{\Cov}{Cov}
\DeclareMathOperator{\MSE}{MSE}
\DeclareMathOperator{\1}{\mathds{1}}
\DeclareMathOperator{\E}{\mathbb{E}}
\DeclareMathOperator{\Prob}{\mathbb{P}}

\newcommand\independent{\protect\mathpalette{\protect\independenT}{\perp}}
\def\independenT#1#2{\mathrel{\rlap{$#1#2$}\mkern2mu{#1#2}}}


\lstset{
  style              = Matlab-editor,
  captionpos         =b,
  basicstyle         = \mlttfamily,
  escapechar         = ",
  mlshowsectionrules = true,
}
\begin{document}
\courseheader



\newcounter{hwcnt}
\setcounter{hwcnt}{1} % set to the times of Homework

\begin{center}
  \underline{\bf Homework 2 } \\
\end{center}
\begin{flushleft}
  YOUR NAME: Guoqing Zhang\hfill
  \today
\end{flushleft}
\hrule

\vspace{2em}
\setlist[enumerate,1]{label=\thehwcnt.\arabic*.}
\setlist[enumerate,2]{label=(\alph*)}
\setlist[enumerate,3]{label=\roman*.}
\setlist[enumerate,4]{label=\greek*)}

\flushleft
\rule{\textwidth}{1pt}
\begin{itemize}
\item {\bf Acknowledgments: \/} 
  This template takes some materials from course CSE 547/Stat 548 of Washington University: \small{\url{https://courses.cs.washington.edu/courses/cse547/17sp/index.html}}.

  If you refer to other materials in your homework, please list here.
\item {\bf Collaborators: \/}
  I finish this homework by myself.But wiki ($\href{https://en.wikipedia.org/wiki/Frobenius_inner_product}{Frobenius inner product}$) and the Andrew Ng's pdf($\href{http://cs229.stanford.edu/notes/cs229-notes2.pdf}{cs229-notes2}$) helped me a lot about solving the third question. I also imitate the format of Gaussian distribution($\sigma \neq 1$) in lecture4 to get the answer. But in fact I don't understand the third one very well. 
\end{itemize}
\rule{\textwidth}{1pt}

\vspace{2em}
2.1. We can always meet the situation that the dimension of $\mathbf{x}$ is bigger than the number of the sample.Or the rank of matrix$\mathbf{X}$ is less than its dimension.In this situation the $X^TX$ is singular. In that case the equation is still right. The proof will be like follows:

In that case, we could find a lot $\theta$ to satisfy the equation:
$\mathbf{y} = \mathbf{X}\theta $ (1) .

If we put (1) into the $\mathbf{X}^T \mathbf{y}$,we will get $\mathbf{X}^T\mathbf{X} \theta $, so of couse the old equation is right.

I don't know the other cases, so the proof may not be strict.
$$$$ 
2.2. (a) Firstly, we unfold the function $\ell$:
  \begin{equation*}
    \begin{aligned}
\ell &= \sum _{i=1}^m \log {P_{y|\mathbf{x}}(y_{(i)}|\mathbf{x}^{(i)})}
\\
 &= \sum _{i=1}^m \log \prod _{l=1} ^k \left(
\frac{exp(\mathbf{\theta}_l^T\mathbf{x}^{(i)}+b_l)}{\sum_{j=1}^k exp(\mathbf{\theta}_j^T\mathbf{x}^{(i)}+b_j)} 
 \right)^{\mathbf{1}\{y^{(i)} = l\}} \\  
    &= \sum_{i=1}^m \sum_{l=1}^k \mathbf{1}\{ y^{(i)}=l\}\log\left( \frac{exp(\mathbf{\theta}_l^T\mathbf{x}^{(i)}+b_l)}{\sum_{j=1}^k exp(\mathbf{\theta}_j^T\mathbf{x}+b_j)} \right)
      \end{aligned}
  \end{equation*}
so,if we want get the derivative of $b_l$,the $\ell$ will can also be wrote like this:

\begin{equation*}
\begin{aligned}
\ell &= \sum_{i=1}^m\left( \mathbf{1}\{y^{(i)} = l\}\log\left( \frac{exp(\mathbf{\theta}_l^T\mathbf{x}^{(i)}+b_l)}{\sum_{j=1}^k exp(\mathbf{\theta}_j^T\mathbf{x}^{(i)}+b_j)} \right) + 
\mathbf{1}\{y^{(i)} \neq l\}\log\left( \frac{exp(\mathbf{\theta}_{y^{(i)}}^T\mathbf{x}^{(i)}+b_{y^{(i)}})}{\sum_{j=1}^k exp(\mathbf{\theta}_j^T\mathbf{x}^{(i)}+b_j)} \right)
\right)\\
&= \sum_{i=1}^m (\mathbf{1}\{y^{(i)} = l\} f_1(b_l)+\mathbf{1}\{y^{(i)} \neq l\}f_2(b_l))
\end{aligned}
\end{equation*}
$$
f_1(b_l) = \log\frac{exp(\mathbf{\theta}_l^T\mathbf{x}^{(i)}+b_l)}{\sum_{j=1}^k exp(\mathbf{\theta}_j^T\mathbf{x}^{(i)}+b_j)},
f_2(b_l) = \log\frac{exp(\mathbf{\theta}_{y^{(i)}}^T\mathbf{x}^{(i)}+b_{y^{(i)}})}{\sum_{j=1}^k exp(\mathbf{\theta}_j^T\mathbf{x}^{(i)}+b_j)}
.$$
so we could get the derivative respectively.
\begin{equation*}
\begin{aligned}
\nabla _{b_l} f_1 &=  \frac{\sum_{j=1}^k exp(\mathbf{\theta}_j^T\mathbf{x}^{(i)}+b_j)}{exp(\mathbf{\theta}_l^T\mathbf{x}^{(i)}+b_l)}\cdot
\frac{exp(\mathbf{\theta}_l^T\mathbf{x}^{(i)}+b_l)\cdot \sum_{j=1}^k exp(\mathbf{\theta}_j^T\mathbf{x}^{(i)}+b_j) - exp(2\cdot(\mathbf{\theta}_l^T\mathbf{x}^{(i)}+b_l))}{(\sum_{j=1}^k exp(\mathbf{\theta}_j^T\mathbf{x}^{(i)}+b_j))^2}\\
&=1 - \frac{exp(\mathbf{\theta}_l^T\mathbf{x}^{(i)}+b_l)}{\sum_{j=1}^k exp(\mathbf{\theta}_j^T\mathbf{x}^{(i)}+b_j)}\\
&= 1 - {P_{y|\mathbf{x}}(l|\mathbf{x}^{(i)})}
\end{aligned}
\end{equation*}
For the case "$y^{(i)} \neq l$",we assume that $y^{(i)} = v$.So the part of $f_2$ will be as follows:
\begin{equation*}
\begin{aligned}
\nabla _{b_l} f_2 &=  \frac{\sum_{j=1}^k exp(\mathbf{\theta}_j^T\mathbf{x}^{(i)}+b_j)}{exp(\mathbf{\theta}_v^T\mathbf{x}^{(i)}+b_v)}\cdot
\frac{- exp(\mathbf{\theta}_l^T\mathbf{x}^{(i)}+b_l)\cdot exp(\mathbf{\theta}_v^T\mathbf{x}^{(i)}+b_v)}{(\sum_{j=1}^k exp(\mathbf{\theta}_j^T\mathbf{x}^{(i)}+b_j))^2}\\
&= - \frac{exp(\mathbf{\theta}_l^T\mathbf{x}^{(i)}+b_l)}{\sum_{j=1}^k exp(\mathbf{\theta}_j^T\mathbf{x}^{(i)}+b_j)}\\
&=  - {P_{y|\mathbf{x}}(l|\mathbf{x}^{(i)})}
\end{aligned}
\end{equation*}

To sum up,we can get:
\begin{equation*}
\begin{aligned}
\nabla _{b_l} \ell = \sum _{i=1}^m(\mathbf{1}\{y_i=l\} - P_{y|\mathbf{x}}(l|\mathbf{x}^{(i)}))
\end{aligned}
\end{equation*}
(b) Because we get the optimal $(b_1,...,b_k)$,so we could know that:
  \begin{equation*}
\begin{aligned}
\nabla _{b_l} \ell = \sum _{i=1}^m(\mathbf{1}\{y_i=l\} - P_{y|\mathbf{x}}(l|\mathbf{x}^{(i)})) = 0
\end{aligned}
\end{equation*}
Because we know that the training dataset should not contain any duplicate samples. So we can get $\hat{P}_{\mathbf{x}}(\mathcal{X}) = \frac 1 m$.Then we get:
  \begin{equation*}
\begin{aligned}
 &\sum _{i=1}^m\mathbf{1}\{y_i=l\} = \sum_{i=1}^m P_{y|\mathbf{x}}(l|\mathbf{x}^{(i)})\\
 &\frac{1}{m} \sum _{i=1}^m\mathbf{1}\{y_i=l\}  = \frac{1}{m} \sum_{i=1}^m P_{y|\mathbf{x}}(l|\mathbf{x}^{(i)})\\
 &\hat{P}_y(l) = \sum_{i=1}^m \frac{1}{m} P_{y|\mathbf{x}}(l|\mathbf{x}^{(i)}) = \sum_{i=1}^m P_{y|\mathbf{x}}(l|\mathbf{x}^{(i)}) \cdot \hat{P}_{\mathbf{x}}(\mathbf{x}_i)
\end{aligned}
\end{equation*}

So:
  \begin{equation*}
\begin{aligned}
\hat{P}_y(l) = \sum_{\mathbf{x} \in \mathcal{X}}P_{y|\mathbf{x}}(l|\mathbf{x})\hat{P}_{\mathbf{x}}(\mathcal{X})
\end{aligned}
\end{equation*}
In some cases, the training dataset may have some duplicate samples. In that case, the set $\mathcal{X}$ 's size will be less than m,but the duplicate samples' $\hat{P}_{\mathbf{x}}(\mathcal{X})$ will be bigger than $\frac 1 m$, because of this offset, $ \sum_{\mathbf{x} \in \mathcal{X}}P_{y|\mathbf{x}}(l|\mathbf{x})\hat{P}_{\mathbf{x}}(\mathcal{X})$ will still be equal to $\sum_{i=1}^m P_{y|\mathbf{x}}(l|\mathbf{x}^{(i)})$, so the conclusion will not be changed. 
$$$$
2.3. Like on the class, firstly I consider the situation that $\sum ^{-1} = \mathbf{I}$, so the probability density function will be as follow:
  \begin{equation*}
\begin{aligned}
p_{\mathbf{y}}(\mathbf{y};\mathbf{\mu}) &= \frac {1}{(2 \pi)^{\frac n 2}} exp(-\frac 1 2 (\mathbf{y} - \mathbf{\mu})^T(\mathbf{y} - \mathbf{\mu}))\\
&= \frac {1}{(2 \pi)^{\frac n 2}} exp(-\frac 1 2 \Vert \mathbf{y} - \mathbf{\mu}\Vert^2)\\
&= \frac {1}{(2 \pi)^{\frac n 2}} exp(-\frac 1 2 \Vert \mathbf{y} \Vert^2 +  \mathbf{y}^T\mathbf{\mu} - \frac 1 2 \Vert \mathbf{\mu}\Vert^2)\\
&=\frac {1}{(2 \pi)^{\frac n 2}} exp(-\frac 1 2 \Vert \mathbf{y} \Vert^2)\cdot exp(\mathbf{y}^T\mathbf{\mu} - \frac 1 2 \Vert \mathbf{\mu}\Vert^2)
\end{aligned}
\end{equation*}
So in this case :
  \begin{equation*}
\begin{aligned}
&\eta = \mathbf{\mu} \\
&b(\mathbf{y}) = \frac {1}{(2 \pi)^{\frac n 2}} exp(-\frac 1 2 \Vert \mathbf{y} \Vert^2)\\
&T(\mathbf{y}) = \mathbf{y}\\
&a(\eta) = \frac 1 2 \mathbf{\mu}^T\mathbf{\mu}
\end{aligned}
\end{equation*}
In general:
\begin{equation*}
\begin{aligned}
p_{\mathbf{y}}(\mathbf{y};\mathbf{\mu},\Sigma) &= \frac {1}{(2 \pi)^{\frac n 2}} exp(-\frac 1 2 (\mathbf{y} - \mathbf{\mu})^T \Sigma^{-1} (\mathbf{y} - \mathbf{\mu}))\\
&= \frac {1}{(2 \pi)^{\frac n 2}} exp(-\frac 1 2  \mathbf{y}^T \Sigma^{-1}\mathbf{y} + \mathbf{y}^T\Sigma^{-1}\mathbf{\mu} - \frac 1 2  \mathbf{\mu}^T \Sigma^{-1}\mathbf{\mu} )\\
&= \frac {1}{(2 \pi)^{\frac n 2}} exp(-\frac 1 2 \mathbf{vec}^T({\Sigma}^{-1})\cdot  \mathbf{vec}(\mathbf{yy}^T) +\mathbf{\mu}^T\Sigma^{-1}\mathbf{y}  -\frac 1 2  \mathbf{\mu}^T \Sigma^{-1}\mathbf{\mu})
\end{aligned}
\end{equation*}
So in general cases :
\begin{equation*}
\begin{aligned}
&\eta = \begin{bmatrix}
u^T \Sigma^{-1}\\
-\frac 1 2 \mathbf{vec}(\Sigma^{-1})
\mathbf{\mu}
\end{bmatrix}_{(n+n^2) \times 1} \\
&b(\mathbf{y}) = \frac {1}{(2 \pi)^{\frac n 2}}\\
&T(\mathbf{y}) =  \begin{bmatrix}
\mathbf{y}\\
\mathbf{vec}(\mathbf{yy}^T)
\mathbf{\mu}
\end{bmatrix}_{(n+n^2) \times 1} \\
&a(\eta) = \frac 1 2  \mathbf{\mu}^T \Sigma^{-1}\mathbf{\mu}
\end{aligned}
\end{equation*}
I am really not sure about the answer. I just watched the wiki about Frobenius inner product.And I saw the operator : $\mathbf{vec}$. The answer is what I guessed to make sure they could dot to get a number.
\end{document}

%%% Local Variables:
%%% mode: latex
%%% TeX-master: t
%%% End:
