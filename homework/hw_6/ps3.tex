% Homework template for Learning from Data
% by Xiangxiang Xu <xiangxiangxu.thu@gmail.com>
% LAST UPDATE: October 8, 2018
\documentclass[a4paper]{article}
\usepackage[T1]{fontenc}
\usepackage{amsmath, amssymb, amsthm}
% amsmath: equation*, amssymb: mathbb, amsthm: proof
\usepackage{moreenum}
\usepackage{mathtools}
\usepackage{url}
\usepackage{enumitem}
\usepackage{bm}
\usepackage{graphicx}
\usepackage{subcaption}
\usepackage{booktabs} % toprule
\usepackage[mathcal]{eucal}
\usepackage{dsfont}
\usepackage[numbered,framed]{matlab-prettifier}
%% Definitions for Learning from Data
%% UPDATE: October 8, 2018 by Xiangxiang 
\newcommand{\theterm}{Fall 2018}

\newcommand{\thecoursename}{
Tsinghua-Berkeley Shenzhen Institute\\
%\vspace*{0.1in}
\textsc{Learning from Data}
}

\newcommand{\courseheader}{
\vspace*{-1in}
\begin{center}
\thecoursename \\
\theterm
\vspace*{0.1in}
\hrule
\end{center}
}
\newcommand{\uc}{\underline{c}}    % c, vec
\newcommand{\uv}{\underline{v}}    % x, vec
\newcommand{\uw}{\underline{w}}    % w, vec
\newcommand{\ux}{\underline{x}}    % x, vec
\newcommand{\uy}{\underline{y}}    % y, vec
\newcommand{\uz}{\underline{z}}    % z, vec
\newcommand{\um}{\underline{m}}    % m, vec
\newcommand{\ut}{\underline{t}}    % t, vec
\newcommand{\bc}{\bm{c}}    % c, vec
\newcommand{\bv}{\bm{v}}    % x, vec
\newcommand{\bw}{\bm{w}}    % w, vec
\newcommand{\bx}{\bm{x}}    % x, vec
\newcommand{\by}{\bm{y}}    % y, vec
\newcommand{\bz}{\bm{z}}    % z, vec
% \newcommand{\bm}{\bm{m}}    % m, vec
\newcommand{\bt}{\bm{t}}    % t, vec

\newcommand{\balpha}{\bm{\alpha}}    % alpha, vec
\newcommand{\bxi}{\bm{\xi}}    % xi, vec


\newcommand{\rvx}{\mathsf{x}}    % x, r.v.
\newcommand{\rvy}{\mathsf{y}}    % y, r.v.
\newcommand{\rvz}{\mathsf{z}}    % z, r.v.
\newcommand{\rvw}{\mathsf{w}}    % w, r.v.
\newcommand{\rvv}{\mathsf{v}}    % v, r.v.
\newcommand{\rvm}{\mathsf{m}}    % m, r.v.
\newcommand{\rvt}{\mathsf{t}}    % t, r.v.
\newcommand{\rvH}{\mathsf{H}}    % H, r.v.
\newcommand{\urvx}{\underline{\mathsf{x}}}    % x, r.v. vec
\newcommand{\urvy}{\underline{\mathsf{y}}}    % y, r.v. vec
\newcommand{\urvz}{\underline{\mathsf{z}}}    % z, r.v. vec
\newcommand{\urvw}{\underline{\mathsf{w}}}    % w, r.v. vec
\newcommand{\urvt}{\underline{\mathsf{t}}}    % t, r.v. vec
\newcommand{\defeq}{\triangleq} %\coloneqq
\newcommand{\reals}{\mathbb{R}}
\newcommand{\T}{\mathrm{T}}    % transpose
\newcommand{\BLS}{\mathrm{BLS}}    % BLS
\newcommand{\LLS}{\mathrm{LLS}}    % LLS
\newcommand{\MVU}{\mathrm{MVU}}    % MVU

\DeclareMathOperator*{\maximize}{maximize}    % maximize
\DeclareMathOperator*{\minimize}{minimize}    % minimize
\newcommand{\st}{\mathrm{subject~to}}    % minimize


% \newcommand{\E}[1]{\mathbb{E}\left[{#1}\right]}
% \newcommand{\Prob}[1]{\mathbb{P}\left({#1}\right)}
\DeclareMathOperator*{\argmax}{arg\,max}
\DeclareMathOperator*{\argmin}{arg\,min}
\DeclareMathOperator*{\argsup}{arg\,sup}
\DeclareMathOperator*{\arginf}{arg\,inf}
\DeclareMathOperator{\Var}{Var}
\DeclareMathOperator{\Cov}{Cov}
\DeclareMathOperator{\MSE}{MSE}
\DeclareMathOperator{\1}{\mathds{1}}
\DeclareMathOperator{\E}{\mathbb{E}}
\DeclareMathOperator{\Prob}{\mathbb{P}}

\newcommand\independent{\protect\mathpalette{\protect\independenT}{\perp}}
\def\independenT#1#2{\mathrel{\rlap{$#1#2$}\mkern2mu{#1#2}}}


\lstset{
  style              = Matlab-editor,
  captionpos         =b,
  basicstyle         = \mlttfamily,
  escapechar         = ",
  mlshowsectionrules = true,
}
\begin{document}
\courseheader



\newcounter{hwcnt}
\setcounter{hwcnt}{1} % set to the times of Homework

\begin{center}
  \underline{\bf Homework 2 } \\
\end{center}
\begin{flushleft}
  YOUR NAME Guoqing Zhang\hfill
  \today
\end{flushleft}
\hrule

\vspace{2em}
\setlist[enumerate,1]{label=\thehwcnt.\arabic*.}
\setlist[enumerate,2]{label=(\alph*)}
\setlist[enumerate,3]{label=\roman*.}
\setlist[enumerate,4]{label=\greek*)}

\flushleft
\rule{\textwidth}{1pt}
\begin{itemize}
\item {\bf Acknowledgments: \/} 
  This template takes some materials from course CSE 547/Stat 548 of Washington University: \small{\url{https://courses.cs.washington.edu/courses/cse547/17sp/index.html}}.

  If you refer to other materials in your homework, please list here.
\item {\bf Collaborators: \/}
  I finish this homework by myself.
\end{itemize}
\rule{\textwidth}{1pt}

\vspace{2em}
3.1 (a) i.
First, it's obvious that we can just focus on the part: $\sum_{x \in C_j} \Vert x - \mu_j\Vert^2$ and $\frac{1}{2\vert C_j \vert} \sum_{x,x'\in C_j}\Vert x - x'\Vert^2$.

$$
\begin{aligned}
\sum_{x \in C_j} \Vert x - \mu_j\Vert^2&= \sum_{x \in C_j} \left(\Vert x\Vert ^2  -2x^T\mu_j +\Vert\mu_j\Vert^2\right)\\
&= \sum_{x \in C_j} \Vert x\Vert^2 - \frac 2{\vert C_j\vert}\sum_{x \in C_j} x^T \sum_{x' \in C_j} x' +\frac{1}{\vert C_j \vert^2}\sum_{x \in C_j}  \sum_{x' in C_j} x'^T \sum_{x'' \in C_j} x''  \\
&= \sum_{x \in C_j} \Vert x\Vert^2  - \frac{2}{\vert C_j\vert} \sum_{x ,x'\in C_j } x^Tx'+ \frac{\vert C_j\vert}{\vert C_j\vert^2} \sum_{x ,x'\in C_j } x^Tx'\\
&= \sum_{x \in C_j} \Vert x\Vert^2  -  \frac{1}{\vert C_j\vert}\sum_{x ,x'\in C_j } x^Tx'
\end{aligned}
$$
$$
\begin{aligned}
\frac{1}{2\vert C_j \vert} \sum_{x,x'\in C_j}\Vert x - x'\Vert^2 &= \frac{1}{2\vert C_j \vert} \sum_{x,x'\in C_j}\left( \Vert x\Vert ^2 - 2x^Tx'+ \Vert x'\Vert^2\right)\\
& = \frac{1}{\vert C_j \vert} \sum_{x,x' \in C_j} \Vert x\Vert^2 - \frac{1}{\vert C_j\vert} \sum_{x,x' \in C_j} x^Tx'\\
&= \sum_{x \in C_j} \Vert x\Vert^2  -  \frac{1}{\vert C_j\vert}\sum_{x ,x'\in C_j } x^Tx'
\end{aligned}
$$
So they are equivalent.

ii. 
Because of i, we could know:
$$
\begin{aligned}
\sum_{j=1}^k\sum_{x \in C_j} \Vert x - \mu_j \Vert^2 &= \sum_{j=1}^k\sum_{x \in C_j} \left(\Vert x\Vert^2  -  \frac{1}{\vert C_j\vert}\sum_{x ,x'\in C_j } x^Tx'\right)\\
&=\sum_{i = 1}^m \Vert x \Vert^2 - \sum_{j=1}^k\frac{1}{\vert C_j\vert}\sum_{x ,x'\in C_j } x^Tx'\\
&= A - \sum_{j=1}^k\frac{1}{\vert C_j\vert}\sum_{x ,x'\in C_j } x^Tx'
\end{aligned}
$$

Where $A=\sum_{i = 1}^m \Vert x \Vert^2$ is a constant.

So $argmin_C (A - \sum_{j=1}^k\frac{1}{\vert C_j\vert}\sum_{x ,x'\in C_j } x^Tx') = argmax_c(\sum_{j=1}^k\frac{1}{\vert C_j\vert}\sum_{x ,x'\in C_j } x^Tx') $

First,let's just focus on the $\sum_{j=1}^k\vert C_i \vert\vert C_j\vert \Vert \mu_i - \mu_j \Vert^2$.
$$
\begin{aligned}
\sum_{j=1}^k\vert C_i \vert\vert C_j\vert \Vert \mu_i - \mu_j \Vert^2 &= \vert C_i \vert \sum_{j=1}^k \vert C_j\vert\left(\Vert \mu_i \Vert^2 - 2\mu_i^T\mu_j + \Vert \mu_j \Vert^2\right)\\
&=  \sum_{j=1}^k  \left(\frac{\vert C_j\vert}{\vert C_i\vert}\sum_{x,x' \in C_i} x^Tx' -2\sum_{x \in C_i, x' \in C_j} x^Tx'  + \frac{\vert C_i\vert}{\vert C_j\vert}\sum_{x,x' \in C_j} x^Tx'\right)
\end{aligned}
$$
Then, combine the whole fomulation:
$$
\begin{aligned}
\sum_{i=1}^k\sum_{j=1}^k\vert C_i \vert\vert C_j\vert \Vert \mu_i - \mu_j \Vert^2 &= \sum_{i=1}^k\sum_{j=1}^k \left(\frac{\vert C_j\vert}{\vert C_i\vert}\sum_{x,x' \in C_i} x^Tx' + \frac{\vert C_i\vert}{\vert C_j\vert}\sum_{x,x' \in C_j} x^Tx'-2\sum_{x \in C_i, x' \in C_j} x^Tx'\right)\\
&= 2\sum_{i=1}^k\sum_{j=1}^k\left(\frac{\vert C_j\vert}{\vert C_i\vert}\sum_{x,x' \in C_i} x^Tx'-\sum_{x \in C_i, x' \in C_j} x^Tx' \right)\\
&= 2\sum_{i=1}^k\frac{m}{\vert C_i\vert}\sum_{x,x' \in C_i} x^Tx'-2\sum_{i=1}^k\sum_{j=1}^k\sum_{x \in C_i, x' \in C_j} x^Tx'\\
&= 2\sum_{i=1}^k \frac{m+\vert C_i \vert}{\vert C_i \vert} \sum_{x,x' \in C_i} x^Tx' - 2\left(\sum_{i=1}^k\sum_{j=1}^k\sum_{x \in C_i, x' \in C_j} x^Tx' +\sum_{i=1}^k\sum_{x,x' \in C_i} x^Tx' \right)\\
&= 2\sum_{i=1}^k \frac{m+\vert C_i \vert}{\vert C_i \vert} \sum_{x,x' \in C_i} x^Tx' - A\\
&= 2\sum_{i=1}^k (\frac{m}{\vert C_i\vert} + 1) \sum_{x,x' \in C_i} x^Tx' - A
\end{aligned}
$$ 

Where A is a constant, it's equal to 	all the pair's product in X. 
Now I have some trouble to eliminate '1'. It's a little embarrassing. I am not sure which step is wrong, or I can't proof it in this way.

I also see something about the law of total variance could help to prove it. But I am not very clear about it, so I won't write down.

(b) i.
If the algorithm has converged, than the $\mu$ would be never change, so it's obvious that the distortion will increase.

The distortion could be compute at two states. First, from x we got $\mu$, compute $J$, then, we reassign the clusters, compute $J$.

It's obvious that in the second state, the $J$ will not increase, because:
$$
J(\{c^{(i)}\}_{i=1}^m,\{\mu_j\}_{j=1}^k) = \sum_{i=1}^m\Vert x^{(i) - \mu_{c^{i}}}\Vert^2
$$

And the x is reassigned to the closest $\mu_j$, which means:$\Vert x^{(i) - \mu'_{c^{(i)}}}\Vert ^2 \leq \Vert x^{(i) - \mu_{c^{(i)}}}\Vert ^2$. Each term will not increase, so the distortion will also not increase.

Then what we want to prove is:
$$
argmin_{p} \sum_{x\in C_j} \Vert x - p \Vert^2 = \mu_j
$$

Assume $l(p) =  \sum_{x\in C_j} \Vert x - p \Vert^2$.
$$
\begin{aligned}
\frac{\partial l}{\partial p} &=\frac{\partial\sum_{x \in C_j} \left(\Vert x \Vert^2 - 2x^Tp + \Vert p\Vert^2\right)}{\partial p}\\
&=\sum_{x \in C_j}2(p -  x)\\
&= 2 \vert C_j\vert p - 2 \sum_{x \in C_j} x
\end{aligned}
$$

Then let $\frac{\partial l}{\partial p} = 0$, we could get:
$$
p = \frac{1}{\vert C_j\vert} \sum_{x \in C_j} x = \mu_j
$$
This means J will not increase in the first state.

The Lloyd's algorithm just iterates the two steps, so the distortion will not increase. 

ii. It will always converge. First, the distortion J have the lower bound:$J \ge 0$. If we consider the valuse of J computed from each iteration as a sequence, we know that Monotonous bounded sequence has a convergence. Frome i, we know that this sequence is monotonous, which means this algorhithm will converge.

$$$$
3.2 (a) i. $$
\begin{aligned}
\mu_T \Cov(x)\mu &= 
\begin{bmatrix}
\mu_1&\mu_2&\cdots&\mu_d 
\end{bmatrix}
\begin{bmatrix}
\Cov(x_1,x_1)&\Cov(x_1,x_2)&\cdots &\Cov(x_1,x_d)\\
\Cov(x_2,x_1)&\Cov(x_2,x_2)&\cdots &\Cov(x_2,x_d)\\
\vdots& \vdots & \ddots &\vdots\\
\Cov(x_d,x_1) &\Cov(x_d,x_2) &\cdots &\Cov(x_d,x_d)
\end{bmatrix}\begin{bmatrix}
\mu_1\\
\mu_2\\
\vdots\\
\mu_d 
\end{bmatrix}\\
&= \begin{bmatrix}
\mu_1&\mu_2&\cdots&\mu_d 
\end{bmatrix}\begin{bmatrix}
D(x_1)&\Cov(x_1,x_2)&\cdots &\Cov(x_1,x_d)\\
\Cov(x_2,x_1)&D(x_2)&\cdots &\Cov(x_2,x_d)\\
\vdots& \vdots & \ddots &\vdots\\
\Cov(x_d,x_1) &\Cov(x_d,x_2) &\cdots &D(x_d)
\end{bmatrix}\begin{bmatrix}
\mu_1\\
\mu_2\\
\vdots\\
\mu_d 
\end{bmatrix}\\
&= \sum_{i=1}^d \mu_i^2D(x_i)+\sum_{i=1}^{d-1}\sum_{j=i+1}^d 2\mu_i \mu_j \Cov(x_i,x_j)\\
&= \sum_{i=1}^d D(\mu_ix_i)+\sum_{i=1}^{d-1}\sum_{j=i+1}^d 2 \Cov(\mu_ix_i,\mu_j x_j)\\
&=  D(\mu_1x_1 + \mu_2x_2+ \cdots \mu_dx_d)\\
& \ge 0
\end{aligned}
$$
We could do this because:
$$
D(x) + D(y) +2\Cov(x,y) = D(x+y),\Cov(x,z)+\Cov(y,z) = \Cov(x+y,z)
$$
So:
$$
\begin{aligned}
&D(x)+D(y) +D(z) +2\Cov(x,y) +2 \Cov(x,z) + 2\Cov(y,z) \\
&= D(x+y) +D(z)+ 2\Cov(x,z) + 2\Cov(y,z)\\
&=D(x+y) +D(z) +2\Cov(x+y,z)\\
&=D(x+y+z)
\end{aligned}
$$

In the same way, we could increase 3 variables to n variables, so until now I finish the proof.

ii. From i, we could get:
$$
\begin{aligned}
tr(\Cov(x)) &= \sum_{i=1}^d  D(x_i)
\end{aligned}
$$
$$
\begin{aligned}
\mathbb{E}[\Vert x - \mathbb{E}[x] \Vert^2] &= \mathbb E[(x_1-\mathbb{E}[x_1])^2 + ... +(x_d-\mathbb{E}[x_d])^2]\\
&= \sum_{i=1}^d \mathbb E [(x_i - \mathbb E[x_i])^2]\\
&= \sum_{i=1}^d D(x_i)
\end{aligned}
$$
So, we get that $ tr(\Cov(x)) =\mathbb{E}[\Vert x - \mathbb{E}[x] \Vert^2]  $.

(b) If we want to the $\hat C$ is non-singular, where $\hat C$ is a $d \times d$ matrix. The intuition is that $m\ge d$. 

And the intuition is true. Because the rank of the matrix $\hat C$ is limited by the rank of $X$, where
$$
X_{m \times d} = \begin{bmatrix}
x_1^T\\
x_2^T\\
\vdots\\
x_m^T
\end{bmatrix}.
$$

So we could see: $r(X) \leq \min(m,d)$. So we could get that $m \ge d$.
\end{document}
%%% Local Variables:
%%% mode: latex
%%% TeX-master: t
%%% End:
